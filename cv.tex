%%%%%%%%%%%%%%%%%%%%%%%%%%%%%%%%%%%%%%%%%
% Medium Length Graduate Curriculum Vitae
% LaTeX Template
% Version 1.1 (9/12/12)
%
% This template has been downloaded from:
% http://www.LaTeXTemplates.com
%
% Original author:
% Rensselaer Polytechnic Institute (http://www.rpi.edu/dept/arc/training/latex/resumes/)
%
% Important note:
% This template requires the res.cls file to be in the same directory as the
% .tex file. The res.cls file provides the resume style used for structuring the
% document.
%
%%%%%%%%%%%%%%%%%%%%%%%%%%%%%%%%%%%%%%%%%

%----------------------------------------------------------------------------------------
%	PACKAGES AND OTHER DOCUMENT CONFIGURATIONS
%----------------------------------------------------------------------------------------

\documentclass[margin, 9pt]{res} % Use the res.cls style, the font size can be changed to 11pt or 12pt here

\usepackage[T1]{fontenc}
\usepackage[sc]{mathpazo} % Use the great palatino font provided by the mathpazo package. Add osf for old style figures

\usepackage[colorlinks = true,
            linkcolor = blue,
            urlcolor  = blue,
            citecolor = blue,
            anchorcolor = blue]{hyperref}
\usepackage{color}
\usepackage{parskip}
\usepackage{enumitem}
\usepackage[T1]{fontenc}

\setlength{\textwidth}{5.1in} % Text width of the document

\begin{document}

%----------------------------------------------------------------------------------------
%	NAME AND ADDRESS SECTION
%----------------------------------------------------------------------------------------

\moveleft.5\hoffset\centerline{\huge\bf Yuhao Zhu} % Your name at the top
\vspace{-5pt}
 
\moveleft\hoffset\vbox{\hrule width\resumewidth height .5pt}\smallskip % Horizontal line after name; adjust line thickness by changing the '1pt'
 
%\moveleft.5\hoffset\centerline{UT Austin ECE}
%\moveleft.5\hoffset\centerline{yzhu@utexas.edu}
%\moveleft.5\hoffset\centerline{\color{blue}{\url{http://yuhaozhu.com/}}}

\begin{minipage}{0.5\linewidth}
  \flushleft
  The University of Texas at Austin \\
  Electrical and Computer Engineering\\
\end{minipage}
\begin{minipage}{0.5\linewidth}
  \flushright
  \texttt{yzhu@utexas.edu} \\
  \color{blue}{\url{http://yuhaozhu.com/}} \\
\end{minipage}

%----------------------------------------------------------------------------------------

\begin{resume}

%----------------------------------------------------------------------------------------
%	RESEARCH INTERESTS SECTION
%----------------------------------------------------------------------------------------

\vspace*{-5pt}
\section{RESEARCH\\ INTERESTS} 

I am interested in building better hardware and software to make next-generation client + cloud computing faster, more energy-efficient, and deliver user satisfaction.

In my dissertation, I am designing a high-performance and energy-efficient mobile computing substrate that incorporates synergistic cross-layer optimizations across the
hardware, software, and application layers to maximize system efficiency for Web applications.
 
%\begin{itemize}[noitemsep, leftmargin=*]
%	\item Programming language support
%	\item Runtime system design
%	\item Processor (micro)architecture.
%\end{itemize}

%\section{OBJECTIVE}
%Apply for summer internship 2015\\

%----------------------------------------------------------------------------------------
%	EDUCATION SECTION
%----------------------------------------------------------------------------------------

%\vspace*{-5pt}
\section{EDUCATION}

\textbf{The University of Texas at Austin} \hfill\textit{2010-present}\\
Ph.D. student, Electrical and Computer Engineering\\
Advisor: \href{http://3nity.io/~vj/}{Vijay Janapa Reddi}\\

\vspace*{-5pt}
\textbf{Beihang University, Beijing, China} \hfill\textit{2006-2010}\\
B.S., Computer Science and Engineering\\
Advisor: \href{https://research.nvidia.com/content/yangdong-steve-deng}{Yangdong Deng}

%----------------------------------------------------------------------------------------
%	PUBLICATIONS SECTION
%----------------------------------------------------------------------------------------
 
%\vspace*{-5pt}
\section{PUBLICATIONS}

{\large\textbf{Journal Articles}}

\begin{itemize}[leftmargin=*] \itemsep 0pt
%\vspace*{-5pt}
	\item \textbf{Yuhao Zhu}, Matthew Halpern, Vijay Janapa Reddi\\
          \href{http://yuhaozhu.com/pubs/ieeemicro15.pdf}{The Role of the CPU in Energy-Efficient Mobile Web Browsing}\\
          \textit{IEEE MICRO Mobile Special Issue}, Jan/Feb 2015, 35(1):26-33

	\item \textbf{Yuhao Zhu}, Aditya Srikanth, Jingwen Leng, Vijay Janapa Reddi\\
          \href{http://yuhaozhu.com/pubs/cal12.pdf}{Exploiting Webpage Characteristics for Energy-Efficient Mobile Web Browsing}\\
          \textit{Computer Architecture Letters (CAL)}, Oct 2012, 13(1):33-36\\
          \textbf{\textsc{One of Three Papers Awarded Best of CAL for 2014}}

	\item \textbf{Yuhao Zhu}, Bo Wang, Yangdong Deng\\
          \href{http://yuhaozhu.com/pubs/todaes11.pdf}{Massively Parallel Logic Simulation with GPUs}\\
          \textit{ACM Transactions on Design Automation of Electronic Systems (TODAES)}, June 2011, 16(3):29
\end{itemize}

\vspace*{7pt}
{\large\textbf{Conference Papers}}

\begin{itemize}[leftmargin=*] \itemsep 0pt
%\vspace*{-5pt}
	\item \textbf{Yuhao Zhu}, Daniel Richins, Matthew Halpern, Vijay Janapa Reddi\\
          \href{}{Microarchitectural Implications of Event-driven Server-side Web Applications}\\
          \textit{MICRO 2015 (to appear)}

	\item \textbf{Yuhao Zhu}, Matthew Halpern, Vijay Janapa Reddi\\
          \href{http://yuhaozhu.com/pubs/hpca15.pdf}{Event-based Scheduling for Energy-Efficient QoS (eQoS) in Mobile Web Applications}\\
          \textit{HPCA 2015}

	\item Matthew Halpern, \textbf{Yuhao Zhu}, Vijay Janapa Reddi\\
          \href{http://yuhaozhu.com/pubs/ispass15.pdf}{Mosaic: Cross-Platform User-Interaction Record and Replay Tool for the Fragmented Android Ecosystem}\\
          \textit{ISPASS 2015}

	\item \textbf{Yuhao Zhu}, Vijay Janapa Reddi\\
          \href{http://yuhaozhu.com/pubs/isca14.pdf}{WebCore: Architectural Support for Mobile Web Browsing}\\
          \textit{ISCA 2014}

	\item Chen Zhou, Xiaofei Wang, Weichao Xu, \textbf{Yuhao Zhu}, Vijay Janapa Reddi, Chris Kim\\
          \href{http://yuhaozhu.com/pubs/irps14.pdf}{Estimation of Instantaneous Frequency Fluctuation in a Fast DVFS Environment Using an Empirical BTI Stress-Relaxation Model}\\
          \textit{IRPS 2014}

	\item \textbf{Yuhao Zhu}, Vijay Janapa Reddi\\
          \href{http://yuhaozhu.com/pubs/hpca13.pdf}{High-Performance and Energy-Efficient Mobile Web Browsing on Big/Little Systems}\\
          \textit{HPCA 2013}

	\item \textbf{Yuhao Zhu}, Yangdong Deng, Yubei Chen\\
          \href{http://yuhaozhu.com/pubs/dac11.pdf}{Hermes: An Integrated CPU/GPU Microarchitecture for IP Routing}\\
          \textit{DAC 2011}

	\item Bo Wang, \textbf{Yuhao Zhu}, Yangdong Deng\\
          \href{http://yuhaozhu.com/pubs/dac10.pdf}{Distributed Time, Conservative Parallel Logic Simulation on GPUs}\\
          \textit{DAC 2010}
\end{itemize}
 
%----------------------------------------------------------------------------------------
%	RESEARCH EXPERIENCE SECTION
%---------------------------------------------------------------------------------------- 

\section{RESEARCH\\ EXPERIENCE}

{\large\textbf{The University of Texas at Austin (\textit{Graduate})}}\\

\vspace*{-7pt}
{\large\textit{High-Performance and Energy-Efficient Web Computing}} \hfill\textit{8/2011-present}\\

\vspace*{-7pt}
Server Microarchitecture Implications of Event-driven Web Applications \hfill(MICRO 2015)\\
\vspace*{-10pt}
\begin{itemize}[leftmargin=*] \itemsep -3pt
\vspace*{-5pt}
	\item Identified single-thread performance as a major bottleneck for event-driven server workloads due to front-end inefficiencies in microarchitecture
	\item Correlated the root-cause of front-end inefficiencies to the event-driven programming model: little intra-event code reuse coupled with heavy inter-event code reuse
	\item Developed a microarchitecture mechanism that coordinates the instruction cache replacement policy with instruction prefetcher to significantly mitigate front-end bottlenecks.
\end{itemize}

\medskip
Understanding the Role of CPU in Mobile Web Browsing \hfill(\textit{IEEE MICRO})\\
\vspace*{-10pt}
\begin{itemize}[leftmargin=*] \itemsep -3pt
\vspace*{-5pt}
	\item Instrumented six top Android smartphones from 2009 to 2014 to quantify how webpage load time and energy consumption vary with CPU performance and network latency
	\item Proposed a new CPU DVFS governor that coordinates CPU performance with net-work latency for energy savings with little performance impact.
\end{itemize}

\medskip
Event-based Scheduling for Energy-Efficient QoS \hfill(\textit{HPCA 2015})\\
\vspace*{-10pt}
\begin{itemize}[leftmargin=*] \itemsep -3pt
\vspace*{-5pt}
	\item Introduced the notion of ``energy-efficient QoS'' (eQoS). It is a framework to systematically reason about energy-QoS trade-offs in mobile Web applications
	%\item Studied fundamental event characteristics (latency and intensity) in mobile Web apps
	\item Proposed and developed event-based scheduling (EBS) as a new method to achieve eQoS. It trades off performance with energy consumption without hurting user QoS experience.
\end{itemize}

\medskip
WebCore: Specialized Processor Architecture for Mobile Web Applications \hfill(\textit{ISCA 2014})\\
\vspace*{-10pt}
\begin{itemize}[leftmargin=*] \itemsep -3pt
\vspace*{-5pt}
	\item Performed microarchitecture design space explorations to identify the proper baseline architecture for energy-efficient mobile Web browsing
	\item Developed specialized hardware accelerators and cache structures that mitigate instruction delivery and data supply bottlenecks in the general purpose design.
\end{itemize}

\medskip
Energy-Efficient Mobile Web Browsing on Big/Little Systems \hfill(\textit{CAL, HPCA 2013})\\
\vspace*{-10pt}
\begin{itemize}[leftmargin=*] \itemsep -3pt
\vspace*{-5pt}
	\item Instrumented the Pandaboard (ARM Cortex-A9 based) and Beagleboard (Cortex-A8 based) development boards for hardware power/energy analysis
	\item Built regression models for predicting webpage load time and energy consumption
	\item Implemented a scheduling mechanism in Mozilla Firefox that leverages big/little heterogeneous systems and outperforms conventional OS DVFS in both performance and energy.
\end{itemize}

%\medskip
%{\large\textit{CPU Microarchitecture} (with Prof. Yale N. Patt)} \hfill\textit{8/2010-7/2011}\\
%
%\vspace*{-7pt}
%Microarchitecture Support for Speculative Loop-level SIMDization\\
%\vspace*{-10pt}
%\begin{itemize}[leftmargin=*] \itemsep -3pt
%\vspace*{-5pt}
%	\item Designed Pintools to characterize multiple benchmarks (SPEC2006, SPLASH, etc.) in order to identify the upper bound of loop level parallelization
%	\item Proposed hardware mechanisms for dynamically vectorizing programs on a cycle-accurate x86 microarchitecture simulator.
%\end{itemize}

\bigskip
{\large\textbf{Tsinghua University (\textit{Undergraduate})}}\\

\vspace*{-7pt}
{\large\textit{GPU Programming and Microarchitecture} (with Prof. Yangdong Deng)} \hfill\textit{6/2009-7/2010}\\

\vspace*{-7pt}
Hermes: An Integrated CPU/GPU Microarchitecture for IP Routing \hfill(\textit{DAC 2011})\\
\vspace*{-10pt}
\begin{itemize}[leftmargin=*] \itemsep -3pt
\vspace*{-5pt}
	\item Proposed a CPU+GPU heterogeneous IP router microarchitecture to adaptively balance packet delay and overall throughput
	\item Developed analytical models to estimate the delay and throughput of IP packets.
\end{itemize}

\medskip
Parallel Gate Level Logical Simulator on GPUs \hfill(\textit{DAC 2010, TODAES})\\
\vspace*{-10pt}
\begin{itemize}[leftmargin=*] \itemsep -3pt
\vspace*{-5pt}
	\item Proposed and implemented dynamic paging into GPGPU programming to mitigate the limited memory capacity of GPUs
	\item Achieved a performance again of 48X for gate-level simulation compared with Synopsys' sequential simulator.
\end{itemize}

%----------------------------------------------------------------------------------------
%	COMMUNITY SERVICES SECTION
%----------------------------------------------------------------------------------------

\section{COMMUNITY\\ SERVICES}

\textbf{Program Committee Member}\\
\vspace*{-10pt}
\begin{itemize}[leftmargin=*] \itemsep -3pt
\vspace*{-5pt}
	\item Tiny Transactions on Computer Science (TinyToCS) Volumes III (\url{http://tinytocs.org/})\par
	\item Artifacts Evaluation CGO-PPoPP 2015\par
\end{itemize}
\textbf{Solicited Reviewer}\\
\vspace*{-10pt}
\begin{itemize}[leftmargin=*] \itemsep -3pt
\vspace*{-5pt}
	\item DAC 2011, 2012\par
	\item TODAES 2011\par
    \item ESL 2015\par
\end{itemize}
\textbf{Emerging Topics in Computer System Design Reading Group} (\url{https://goo.gl/gzT4Fw})
Co-founder and regular discussion moderator. Organized the weekly discussion seminar on computer system research within UT ECE (\textasciitilde 15 graduate students.)

%----------------------------------------------------------------------------------------
%	INDUSTRIAL EXPERIENCE SECTION
%----------------------------------------------------------------------------------------

\section{INDUSTRIAL\\ EXPERIENCE} 

{\textbf{Google Inc.}} \hfill{Summer 2015}\\
\vspace*{-10pt}
\begin{itemize}[leftmargin=*] \itemsep -3pt
\vspace*{-5pt}
	\item Led the efforts of \href{https://docs.google.com/document/d/1La_0PPfsTqHJihazYhff96thhjPtvq1KjAUOJu0dvEg/edit#heading=h.46jecek204ph}{Flow API} Version 2 as part of Trace Event, which is a programming model for Chrome code instrumentation and analysis. It is incorporated into Chrome.
	\item Developed analysis infrastructure that associates Chrome runtime behaviors with user inputs to enable critical path analysis. It is incorporated into \href{https://github.com/catapult-project/catapult}{Catapult}, which is the official performance analysis tool for Chrome.
\end{itemize}

\medskip
{\textbf{AMD Research Lab}} \hfill{Summer 2013}\\
\vspace*{-10pt}
\begin{itemize}[leftmargin=*] \itemsep -3pt
\vspace*{-5pt}
	\item Studied the Web browser caching mechanism in Chromium
	\item Developed performance analytical models for Web browser caching.
\end{itemize}

\medskip
{\textbf{AMD Research Lab}} \hfill{Summer 2012}\\
\vspace*{-10pt}
\begin{itemize}[leftmargin=*] \itemsep -3pt
\vspace*{-5pt}
	\item Analyzed microarchitecture bottlenecks of Web browsing on AMD Trinity APUs
	\item Added support for AMD Trinity APU in libpfm and PAPI performance counter library.
\end{itemize}

\medskip
{\textbf{STMicroelectronics}} \hfill{Summer 2011}\\
\vspace*{-10pt}
\begin{itemize}[leftmargin=*] \itemsep -3pt
\vspace*{-5pt}
	\item Fully integrated an ARM System-on-Chip (SoC) functional simulator with a set of embedded benchmarks (e.g., EEMBC, Whetstone, etc.)
	\item Designed a timing simulator for the cache hierarchy in an ARM SoC.
\end{itemize}

%----------------------------------------------------------------------------------------
%	TEACHING EXPERIENCE SECTION
%----------------------------------------------------------------------------------------

\section{TEACHING\\ EXPERIENCE} 

{\large{Teaching Assistant}}\\
\vspace*{-5pt}
\begin{itemize}[leftmargin=*] \itemsep -2pt
\vspace*{-5pt}
	\item Lower Division, Introduction to Embedded Systems, with Prof. Jonathan W. Valvano
	\item Upper Division, Computer Architecture, with Prof. Yale N. Patt
	\item Graduate Level, Dynamic Compilation, with Prof. Vijay Janapa Reddi
\end{itemize}

%----------------------------------------------------------------------------------------
%	COURSEWORK SECTION
%----------------------------------------------------------------------------------------

\section{COURSEWORK} 

\vspace*{-2pt}
\begin{itemize}[leftmargin=*] \itemsep -2pt
	\item Computer Architecture (with Prof. Mattan Erez)
	\item Microarchitecture (with Prof. Yale N. Patt)
	\item Parallel Computer Architecture (with Prof. Derek Chiou)
	\item Memory Management (with Prof. Kathryn S McKinley)
	\item Advanced Complier (with Prof. Keshav Pingali)
	\item Dynamic Compilation (with Prof. Vijay Janapa Reddi)
	\item VLSI (with Prof. Jacob Abraham)
\end{itemize}
%----------------------------------------------------------------------------------------
\end{resume}
\end{document}
