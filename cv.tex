%%%%%%%%%%%%%%%%%%%%%%%%%%%%%%%%%%%%%%%%%
% Medium Length Graduate Curriculum Vitae
% LaTeX Template
% Version 1.1 (9/12/12)
%
% This template has been downloaded from:
% http://www.LaTeXTemplates.com
%
% Original author:
% Rensselaer Polytechnic Institute (http://www.rpi.edu/dept/arc/training/latex/resumes/)
%
% Important note:
% This template requires the res.cls file to be in the same directory as the
% .tex file. The res.cls file provides the resume style used for structuring the
% document.
%
%%%%%%%%%%%%%%%%%%%%%%%%%%%%%%%%%%%%%%%%%

%----------------------------------------------------------------------------------------
%	PACKAGES AND OTHER DOCUMENT CONFIGURATIONS
%----------------------------------------------------------------------------------------

\documentclass[margin, 9pt]{res} % Use the res.cls style, the font size can be changed to 11pt or 12pt here

\usepackage[T1]{fontenc}
\usepackage[sc]{mathpazo} % Use the great palatino font provided by the mathpazo package. Add osf for old style figures

\usepackage[colorlinks = true,
            linkcolor = blue,
            urlcolor  = blue,
            citecolor = blue,
            anchorcolor = blue]{hyperref}
\usepackage{color}
\usepackage{parskip}
\usepackage{enumitem}
\usepackage[T1]{fontenc}
\usepackage{filemod}

\setlength{\textwidth}{5.1in} % Text width of the document

\begin{document}

%----------------------------------------------------------------------------------------
%	NAME AND ADDRESS SECTION
%----------------------------------------------------------------------------------------

% \moveleft.5\hoffset\centerline{\huge\bf Curriculum Vitae/Resume}
% \vspace{5pt}
\moveleft.5\hoffset\centerline{\huge\textbf{Ting-Wu Chin}}
% \vspace{5pt}
% \moveleft.5\hoffset\centerline{\bf CSE PhD Fall 2017}
% \vspace{5pt}
% \moveleft.5\hoffset\centerline{\bf U-M ID: 61825190}
\vspace{-5pt}
\moveleft\hoffset\vbox{\hrule width\resumewidth height .5pt}\smallskip % Horizontal line after name; adjust line thickness by changing the '1pt'
\hfill{Last modified \filemodprintdate{\jobname}}




%\moveleft.5\hoffset\centerline{UT Austin ECE}
%\moveleft.5\hoffset\centerline{yzhu@utexas.edu}
%\moveleft.5\hoffset\centerline{\color{blue}{\url{http://yuhaozhu.com/}}}

% \begin{minipage}{0.5\linewidth}
%   \flushleft
%   National Chiao Tung University\\
%   Computer Science\\
% \end{minipage}
% \begin{minipage}{0.5\linewidth}
%   \flushright
%   \texttt{bb1168kk@gmail.com} \\
% \end{minipage}

%----------------------------------------------------------------------------------------

\begin{resume}

%----------------------------------------------------------------------------------------
%	RESEARCH INTERESTS SECTION
%----------------------------------------------------------------------------------------

\vspace*{-5pt}
\section{RESEARCH\\ INTERESTS}

I am interested in the intersection among applied machine learning, computer
vision, and software runtime.

%\section{OBJECTIVE}
%Apply for summer internship 2015\\

%----------------------------------------------------------------------------------------
%	EDUCATION SECTION
%----------------------------------------------------------------------------------------

%\vspace*{-5pt}
\section{EDUCATION}

\textbf{Carnegie Mellon University} \hfill\textit{2017-present}\\
Ph.D., Electrical and Computer Engineering\\

\textbf{National Chiao Tung University} \hfill\textit{2015-2017}\\
M.S., Computer Science\\
Advisor: Shiao-Li Tsao\\
Cum. GPA: 4.3 / 4.3\\

\vspace*{-5pt}
\textbf{National Chiao Tung University} \hfill\textit{2011-2015}\\
B.S., Computer Science\\
Cum. GPA: 4.0 / 4.3\\

%----------------------------------------------------------------------------------------
%	HONOR SECTION
%----------------------------------------------------------------------------------------

\section{HONORS \&\\ RECOGNITIONS}

\vspace*{-1pt}
\begin{itemize}[leftmargin=*] \itemsep -4pt
  \item MediaTek Domestic PhD Fellowship (One of the five students in Taiwan), 2016
  \item Award of Outstanding Teaching Assistant, 2016
	\item NCTU PhD Fellowship, 2015
	\item CS MSc fall 2015 admission of EPFL, Switzerland, 2015
	\item Outstanding Award, System and Architecture Talent Incubation and
  Training Program, Ministry of Education Republic of China (Taiwan), 2014
	\item Selected Scholarship, System and Architecture Talent Incubation and
  Training Program, Ministry of Education Republic of China (Taiwan), 2014
	\item Outstanding Academic Scholarship in Programing Language course, 2012
\end{itemize}

%----------------------------------------------------------------------------------------
%	PUBLICATIONS SECTION
%----------------------------------------------------------------------------------------

%\vspace*{-5pt}
\section{PUBLICATIONS}

{\large\textbf{Conference Papers}}

\begin{itemize}[leftmargin=*] \itemsep 0pt
%\vspace*{-5pt}
	\item \textbf{Ting-Wu Chin}, Shiao-Li Tsao, Kou-Wei Hung, and Pei-Shu Huang\\[2pt]
          Improving Accuracy of Leakage Power Estimation of Embedded CPUs\\
          \textit{DATE 2017}
	\item Hasan Genc, \textbf{Ting-Wu Chin}, Matthew Halpern, and Vijay Janapa Reddi\\[2pt]
          Optimizing Sensor-Cloud Architectures For Real-Time Autonomous Drone Operation\\
          \textit{Sensors to Cloud Architecture Workshop (SCAW) 2017}
\end{itemize}

{\large\textbf{Journal Papers}}

\begin{itemize}[leftmargin=*] \itemsep 0pt
%\vspace*{-5pt}
	\item \textbf{Ting-Wu Chin}, Chia-Lin Yu, Matthew Halpern, Hasan Genc, Shiao-Li Tsao, and Vijay Janapa Reddi\\[2pt]
          Domain Specific Approximation for Object Detection in Cognitive Visual Agents\\
          \textit{Submitted to IEEE Micro, Special Issue: Approximate Computing}
	\item Hasan Genc, Yazhou Zhu, \textbf{Ting-Wu Chin}, Matthew Halpern, and Vijay Janapa Reddi\\[2pt]
          Flying IoT: Toward a Sensor-Cloud Architecture for Low-Power Cognitive Computer Vision in the Sky\\
          \textit{Submitted to IEEE Micro, Special Issue: Ultra-Low-Power Processors}
\end{itemize}

%----------------------------------------------------------------------------------------
%	RESEARCH EXPERIENCE SECTION
%----------------------------------------------------------------------------------------

\section{RESEARCH\\ EXPERIENCE}

{\large\textbf{National Chiao Tung University (\textit{Graduate})}}\\


{\large\textit{Domain Specific Approximation for Object Detection in Flying Agents} (together with Prof. Vijay Janapa Reddi)} \hfill\textit{8/2016-present}\\
\vspace*{-7pt}
\begin{itemize}[leftmargin=*] \itemsep -3pt
\vspace*{-5pt}
  \item To improve performance for object detection-based flying agents, e.g.
  autonomous drones.
  \item Found that state-of-the-art object detection algorithms such as
  \textit{Faster R-CNN} and \textit{R-FCN} have no more then 2 fps running on
  modern embedded GPU platform, nVidia Tegra X1.
	\item Found that domain-specific approximation techniques such as image size
  scaling and number of proposal scaling can bring up to 7x speed improvement
  without losing accuracy in oracle when the techniques are applied dynamically.
  \item Based on nVidia's Tegra X1, and use OpenCV and Caffe. Counted on NI DAQ
  for power measurement.
\end{itemize}
\vspace*{-7pt}

\medskip

{\large \textbf{\textit{Leakage Power Estimation}} \hfill(DATE 2017, Interactive Presentation)}\\
\vspace*{-10pt}
\begin{itemize}[leftmargin=*] \itemsep -3pt
\vspace*{-5pt}
  \item To improve accuracy of core-level leakage power estimation of embedded
  processors.
  \item We found that estimating leakage power with on-chip thermal sensor is
  inaccurate for certain type of benchmarks.
  \item Proposed both hardware and software methodologies to improve the
  accuracy of leakage power estimation. Reduce the error rate from up to 23\% to
  no more than 3.4\%.
  \item Simulation with Gem5, HotSpot, and McPAT. Real measurement with
  ODROID-XU3 (Samsung Exynos 5420) and infrared camera.
\end{itemize}

\bigskip

{\large\textbf{The University of Texas at Austin (\textit{Visiting Student})}}\\

\vspace*{-7pt}
{\large\textit{Flying IoT} (with Prof. Vijay Janapa Reddi)} \hfill\textit{8/2016-10/2016}\\
\vspace*{-7pt}
\begin{itemize}[leftmargin=*] \itemsep -3pt
\vspace*{-5pt}
  \item To understand and build the next generation software/hardware
  architecture for intelligent autonomous drones, or flying IoT.
  \item Characterize the performance of cognitive workloads on existing sensor
  and cloud compute paradigm.
  \item We found that performance, especially for flying IoT, is important since
  we are not there yet reaching the real-time performance and computing speed
  affect the time to finish a task, which has great impact on power consumption
  due to the time of hovering/flying.
  \item Based on 3DR Solo drone, nVidia Tegra X1, Intel Aero Board, and AirSim.
\end{itemize}
%----------------------------------------------------------------------------------------
%	INDUSTRIAL EXPERIENCE SECTION
%----------------------------------------------------------------------------------------

\section{INDUSTRIAL\\ EXPERIENCE}

% {\textbf{VMFive}} \hfill{Summer 2015}\\
% \vspace*{-10pt}
% \begin{itemize}[leftmargin=*] \itemsep -3pt
% \vspace*{-5pt}
% \item Investigate the implementation of Cells, an containter for smartphone, and
% port it to Android Emulator.
% \item Ported an existing Android container, Cells by SSL Columbia, to the
% goldfish to support Android Emulator.
% \item Basic isolation is done which enables Android Emulator to run more than
% one containers, and it is open source on
% Github(\url{https://github.com/RudyChin/cells})
% \end{itemize}
{\textbf{AILabs.tw}} \hfill{summer 2017}\\
{\textit{Intern, Software Engineer}}
\begin{itemize}[leftmargin=*] \itemsep -3pt
\vspace*{-5pt}
  \item Built up Face Swap, which is an open source project, and wrap up a docker
  container for it.
	\item Built Python wrapper for Face Swap.
	\item Improve the performance of Face Swap by 5x counting on approximation
  techniques.
\end{itemize}

\medskip
{\textbf{MediaTek}} \hfill{Summer 2014}\\
{\textit{Intern, Software Engineer}}
\begin{itemize}[leftmargin=*] \itemsep -3pt
\vspace*{-5pt}
  \item Make the testcases more robust and help develop new API.
	\item Developed the DAG API for supporting implementation of Camera API level
  3 in Hardware Abstraction Layer of Android Open Source Project.
	\item Integrated google test framework into their sanity checking system.
\end{itemize}

%----------------------------------------------------------------------------------------
%	COURSEWORK SECTION
%----------------------------------------------------------------------------------------

\section{COURSE WORK}
{\large\textbf{Graduate}}\\
\vspace*{-10pt}
\begin{itemize}[leftmargin=*] \itemsep -3pt
\vspace*{-5pt}
\item Operating System Design and Implementation
\item Computer Architecture
\item Advance Unix Programming
\item Parallel Programming in Clouds
\end{itemize}
\textbf{Cum. GPA for above courses: 4.3 / 4.3 }

\medskip
{\large\textbf{Undergraduate}}\\
\vspace*{-10pt}
\begin{itemize}[leftmargin=*] \itemsep -3pt
\vspace*{-5pt}
\item Operating System
\item Intro. to Compiler Design
\item Intro. to Embedded System Design and Implementation
\item Microprocessor System Lab.
\item Assembly Language and System Programming
\end{itemize}
\textbf{Cum. GPA for above courses: 4.3 / 4.3 }

%----------------------------------------------------------------------------------------
%	TA SECTION
%----------------------------------------------------------------------------------------
\section{TEACHING EXPERIENCE}
{\large\textbf{Teaching Assistant}}\\
\vspace*{-10pt}
\begin{itemize}[leftmargin=*] \itemsep -3pt
\vspace*{-5pt}
\item Computer Organization, with Prof. Kai-Chiang Wu, S17
\item Virtual Machine, with Prof. Wei Hsu, S17
\item Operating System Design and Implementation, with Prof. Shiao-Li Tsao, S17, S16
\item Microprocessor System Lab., with Prof. Shiao-Li Tsao, F17, F16
\end{itemize}

%----------------------------------------------------------------------------------------
%	MENTOR SECTION
%----------------------------------------------------------------------------------------
\medskip
{\large\textbf{Students Mentored}}\\
\vspace*{-10pt}
\begin{itemize}[leftmargin=*] \itemsep -3pt
\vspace*{-5pt}
\item Te-Chi Chen, Undergraduate Student at National Chiao Tung University
\item Hasan Genc, Undergraduate Student UT Austin
\item Min-Yu Liu, Master Student at National Chiao Tung University
\item Guan-Ren Chen, Master Student at National Chiao Tung University
\end{itemize}

\end{resume}
\end{document}
